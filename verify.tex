\documentclass[11pt]{article}
\usepackage[margin=1in]{geometry}
\usepackage{amsmath}
\usepackage{amssymb}
\usepackage{graphicx}
\usepackage{array}
\usepackage{booktabs}
\usepackage{float}
\usepackage{listings}
\usepackage{xcolor}
\usepackage{parskip}
\usepackage{titlesec}

% Configure listings for Mathematica code
\lstset{
  language=Mathematica,
  basicstyle=\small\ttfamily,
  keywordstyle=\color{blue},
  commentstyle=\color{green!60!black},
  stringstyle=\color{red},
  numbers=none,
  frame=single,
  breaklines=true,
  backgroundcolor=\color{gray!10},
  captionpos=b,
  aboveskip=10pt,
  belowskip=10pt
}

% Adjust section spacing
\titlespacing*{\section}{0pt}{12pt plus 4pt minus 2pt}{6pt plus 2pt minus 2pt}
\titlespacing*{\subsection}{0pt}{10pt plus 3pt minus 2pt}{4pt plus 2pt minus 2pt}

\begin{document}

\title{HW 8-1 pb 5-all}
\date{\today}
\maketitle

\bigskip

\noindent\textbf{Input:}
\begin{lstlisting}
ClearAll [ "\<Global`*\>" ] \
(* Set parameters *) \
a = 1 ; (* length parameter *) g = 1 ; (* potential strength , in units of [HBar]\.b2 / ( ma ) *) \
Print [ Style [ "\<Linear Potential V(x) = g|x| - Rayleigh-Ritz Method\>" , , 16 ] ] \
\
(* === === === = STEP 1 : Define Basis Functions === === === = *) Print [ Style [ "\<STEP 1: Basis Functions\>" , , 14 ] ] f1 [ x_ ] := Exp [ - x ^ 2 / a ^ 2 ] f2 [ x_ ] := x * Exp [ - x ^ 2 / a ^ 2 ] \
Print [ "\<f\:2081(x) = exp(-x\.b2/a\.b2) [EVEN function]\>" ] Print [ "\<f\:2082(x) = x[CenterDot]exp(-x\.b2/a\.b2) [ODD function]\>" ] Print [ ] \
(* Plot basis functions - BLACK AND WHITE *) \
Plot [ { f1 [ x ] , f2 [ x ] } , { x , - 3 , 3 } , PlotStyle -> { { Black , Thick } , (* f1 : solid black *) { Black , Dashed , Thick } (* f2 : dashed black *) } , PlotLegends -> Placed [ LineLegend [ { Graphics [ { Black , Thick , Line [ { { 0 , 0 } , { 1 , 0 } } ] } ] , Graphics [ { Black , Dashed , Thick , Line [ { { 0 , 0 } , { 1 , 0 } } ] } ] } , { "\<f\:2081(x) [solid]\>" , "\<f\:2082(x) [dashed]\>" } ] , { Right , Top } ] , PlotLabel -> Style [ "\<Basis Functions\>" , ] , AxesLabel -> { Style [ "\<x\>" , 12 ] , Style [ "\<f(x)\>" , 12 ] } , GridLines -> Automatic , GridLinesStyle -> Directive [ Gray , Dotted ] , ImageSize -> Large , Frame -> True , FrameStyle -> Black ] \
(* === === === = STEP 2 : Overlap Matrix S === === === = *) \
Print [ Style [ , , 14 ] ] Print [ ] \
S11 = Integrate [ f1 [ x ] ^ 2 , { x , - Infinity , Infinity } , Assumptions -> a > 0 ] Print [ "\<S\:2081\:2081 = [Integral] f\:2081\.b2 dx = \>" , S11 ] Print [ "\< = \>" , N [ S11 , 6 ] ] Print [ ] \
S12 = Integrate [ f1 [ x ] * f2 [ x ] , { x , - Infinity , Infinity } , Assumptions -> a > 0 ] Print [ , S12 , "\< (odd integrand [RightArrow] 0)\>" ] Print [ ] \
S21 = S12 ; Print [ "\<S\:2082\:2081 = S\:2081\:2082 = \>" , S21 ] Print [ ] \
S22 = Integrate [ f2 [ x ] ^ 2 , { x , - Infinity , Infinity } , Assumptions -> a > 0 ] Print [ "\<S\:2082\:2082 = [Integral] f\:2082\.b2 dx = \>" , S22 ] Print [ "\< = \>" , N [ S22 , 6 ] ] Print [ ] \
(* Construct overlap matrix *) \
Smatrix = { { S11 , S12 } , { S21 , S22 } } ; Print [ "\<Overlap Matrix S (symbolic):\>" ] Print [ MatrixForm [ Smatrix ] ] Print [ ] Print [ "\<Overlap Matrix S (numerical):\>" ] Print [ MatrixForm [ N [ Smatrix , 6 ] ] ] Print [ ] \
(* === === === = STEP 3 : Kinetic Energy Matrix T === === === = *) \
Print [ Style [ , , 14 ] ] Print [ ] \
(* Calculate second derivatives *) \
f1pp [ x_ ] = D [ f1 [ x ] , { x , 2 } ] f2pp [ x_ ] = D [ f2 [ x ] , { x , 2 } ] \
Print [ "\<d\.b2f\:2081/dx\.b2 = \>" , f1pp [ x ] ] Print [ ] Print [ "\<d\.b2f\:2082/dx\.b2 = \>" , f2pp [ x ] ] Print [ ] \
T11 = - 1 / 2 * Integrate [ f1 [ x ] * f1pp [ x ] , { x , - Infinity , Infinity } , Assumptions -> a > 0 ] Print [ , T11 ] Print [ "\< = \>" , N [ T11 , 6 ] ] Print [ ] \
T12 = - 1 / 2 * Integrate [ f1 [ x ] * f2pp [ x ] , { x , - Infinity , Infinity } , Assumptions -> a > 0 ] Print [ , T12 , "\< (parity [RightArrow] 0)\>" ] Print [ ] \
T21 = T12 ; Print [ "\<T\:2082\:2081 = T\:2081\:2082 = \>" , T21 ] Print [ ] \
T22 = - 1 / 2 * Integrate [ f2 [ x ] * f2pp [ x ] , { x , - Infinity , Infinity } , Assumptions -> a > 0 ] Print [ , T22 ] Print [ "\< = \>" , N [ T22 , 6 ] ] Print [ ] \
(* Construct kinetic energy matrix *) \
Tmatrix = { { T11 , T12 } , { T21 , T22 } } ; Print [ "\<Kinetic Energy Matrix T (symbolic):\>" ] Print [ MatrixForm [ Tmatrix ] ] Print [ ] Print [ "\<Kinetic Energy Matrix T (numerical):\>" ] Print [ MatrixForm [ N [ Tmatrix , 6 ] ] ] Print [ ] \
(* === === === = STEP 4 : Potential Energy Matrix V === === === = *) \
Print [ Style [ , , 14 ] ] Print [ ] \
V11 = g * Integrate [ Abs [ x ] * f1 [ x ] ^ 2 , { x , - Infinity , Infinity } , Assumptions -> a > 0 ] Print [ , V11 ] Print [ "\< = \>" , N [ V11 , 6 ] ] Print [ ] \
V12 = g * Integrate [ Abs [ x ] * f1 [ x ] * f2 [ x ] , { x , - Infinity , Infinity } , Assumptions -> a > 0 ] Print [ , V12 , "\< (parity [RightArrow] 0)\>" ] Print [ ] \
V21 = V12 ; Print [ "\<V\:2082\:2081 = V\:2081\:2082 = \>" , V21 ] Print [ ] \
V22 = g * Integrate [ Abs [ x ] * f2 [ x ] ^ 2 , { x , - Infinity , Infinity } , Assumptions -> a > 0 ] Print [ , V22 ] Print [ "\< = \>" , N [ V22 , 6 ] ] Print [ ] \
(* Construct potential energy matrix *) \
Vmatrix = { { V11 , V12 } , { V21 , V22 } } ; Print [ "\<Potential Energy Matrix V (symbolic):\>" ] Print [ MatrixForm [ Vmatrix ] ] Print [ ] Print [ "\<Potential Energy Matrix V (numerical):\>" ] Print [ MatrixForm [ N [ Vmatrix , 6 ] ] ] Print [ ] \
(* === === === = STEP 5 : Hamiltonian Matrix H = T + V === === === = *) \
Print [ Style [ "\<STEP 5: Hamiltonian Matrix H = T + V\>" , , 14 ] ] Print [ ] \
Hmatrix = Tmatrix + Vmatrix ; Print [ "\<Hamiltonian Matrix H (symbolic):\>" ] Print [ MatrixForm [ Simplify [ Hmatrix ] ] ] Print [ ] Print [ "\<Hamiltonian Matrix H (numerical):\>" ] Print [ MatrixForm [ N [ Hmatrix , 6 ] ] ] Print [ ] \
Print [ Style [ "\<*** KEY OBSERVATION: H and S are BLOCK-DIAGONAL! ***\>" , ] ] Print [ "\<All off-diagonal elements are ZERO due to parity symmetry.\>" ] Print [ "\<Even function f\:2081 doesn't mix with odd function f\:2082.\>" ] Print [ ] \
(* === === === = STEP 6 : Solve Generalized Eigenvalue Problem === === === = *) \
Print [ Style [ "\<STEP 6: Solve HC = SCE\>" , , 14 ] ] Print [ ] \
(* Numerical solution *) \
{ evalues , evectors } = Eigensystem [ { N [ Hmatrix ] , N [ Smatrix ] } ] ; sortedIndices = Ordering [ evalues ] ; evalues = evalues [ [ sortedIndices ] ] ; evectors = evectors [ [ sortedIndices ] ] ; \
Print [ "\<Rayleigh-Ritz Energy Eigenvalues:\>" ] Print [ "\<E\:2081 = \>" , NumberForm [ evalues [ [ 1 ] ] , 6 ] ] Print [ "\<E\:2082 = \>" , NumberForm [ evalues [ [ 2 ] ] , 6 ] ] Print [ ] \
Print [ "\<Eigenvectors (coefficients [c\:2081, c\:2082]):\>" ] Print [ "\<Ground state: c = [\>" , NumberForm [ evectors [ [ 1 , 1 ] ] , 4 ] , "\<, \>" , NumberForm [ evectors [ [ 1 , 2 ] ] , 4 ] , "\<]\>" ] Print [ "\<Excited state: c = [\>" , NumberForm [ evectors [ [ 2 , 1 ] ] , 4 ] , "\<, \>" , NumberForm [ evectors [ [ 2 , 2 ] ] , 4 ] , "\<]\>" ] Print [ ] \
(* === === === = STEP 7 : Analytical Formulas === === === = *) \
Print [ Style [ "\<STEP 7: Analytical Energy Formulas (Block-Diagonal)\>" , , 14 ] ] Print [ ] \
Print [ "\<Since H and S are block-diagonal, eigenvalues are:\>" ] Print [ ] E1analytical = Hmatrix [ [ 1 , 1 ] ] / Smatrix [ [ 1 , 1 ] ] ; E2analytical = Hmatrix [ [ 2 , 2 ] ] / Smatrix [ [ 2 , 2 ] ] ; \
Print [ "\<E\:2081 = H\:2081\:2081/S\:2081\:2081 = \>" , Simplify [ E1analytical ] ] Print [ "\< = \>" , N [ E1analytical , 6 ] ] Print [ ] Print [ "\<E\:2082 = H\:2082\:2082/S\:2082\:2082 = \>" , Simplify [ E2analytical ] ] Print [ "\< = \>" , N [ E2analytical , 6 ] ] Print [ ] \
(* === === === = STEP 8 : Compare with Exact Results === === === = *) \
Print [ Style [ "\<STEP 8: Comparison with Exact (Analytic) Results\>" , , 14 ] ] Print [ ] \
Print [ "\<For the linear potential V(x) = g|x|, the exact solution\>" ] Print [ "\<involves Airy functions. The energy eigenvalues are:\>" ] Print [ ] Print [ "\< E\:2099 = g^(2/3) [CenterDot] |[Alpha]\:2099|\>" ] Print [ ] Print [ ] Print [ ] \
(* Zeros of Airy function Ai ( - z ) - these are negative of the usual zeros *) \
airyZeros = { 2.33810741 , 4.08794944 , 5.52055983 } ; \
(* For g = 1 , the exact energies are *) \
exactE1 = g ^ ( 2 / 3 ) * airyZeros [ [ 1 ] ] ; exactE2 = g ^ ( 2 / 3 ) * airyZeros [ [ 2 ] ] ; exactE3 = g ^ ( 2 / 3 ) * airyZeros [ [ 3 ] ] ; \
Print [ "\<Exact energy levels (from Airy function):\>" ] Print [ "\<E\:2081(exact) = \>" , NumberForm [ exactE1 , 6 ] ] Print [ "\<E\:2082(exact) = \>" , NumberForm [ exactE2 , 6 ] ] Print [ "\<E\:2083(exact) = \>" , NumberForm [ exactE3 , 6 ] ] Print [ ] \
Print [ Style [ "\<COMPARISON TABLE:\>" , , 12 ] ] Print [ StringRepeat [ "\<-\>" , 72 ] ] Print [ Style [ StringForm [ "\<`` `` `` ``\>" , StringPadRight [ "\<Level\>" , 8 ] , StringPadRight [ "\<Rayleigh-Ritz\>" , 16 ] , StringPadRight [ "\<Exact\>" , 16 ] , "\<Error (%)\>" ] , ] ] Print [ StringRepeat [ "\<-\>" , 72 ] ] \
err1 = 100 * Abs [ evalues [ [ 1 ] ] - exactE1 ] / exactE1 ; err2 = 100 * Abs [ evalues [ [ 2 ] ] - exactE2 ] / exactE2 ; \
Print [ StringForm [ "\<`` `` `` ``\>" , StringPadRight [ "\<E\:2081\>" , 8 ] , StringPadRight [ ToString [ NumberForm [ evalues [ [ 1 ] ] , 6 ] ] , 16 ] , StringPadRight [ ToString [ NumberForm [ exactE1 , 6 ] ] , 16 ] , NumberForm [ err1 , { 5 , 2 } ] ] ] \
Print [ StringForm [ "\<`` `` `` ``\>" , StringPadRight [ "\<E\:2082\>" , 8 ] , StringPadRight [ ToString [ NumberForm [ evalues [ [ 2 ] ] , 6 ] ] , 16 ] , StringPadRight [ ToString [ NumberForm [ exactE2 , 6 ] ] , 16 ] , NumberForm [ err2 , { 5 , 2 } ] ] ] \
Print [ StringRepeat [ "\<-\>" , 72 ] ] Print [ ] \
(* === === === = STEP 9 : Visualization === === === = *) \
Print [ Style [ "\<STEP 9: Visualization of Results\>" , , 14 ] ] Print [ ] \
(* Construct approximate wavefunctions *) \
psi1 [ x_ ] := evectors [ [ 1 , 1 ] ] * f1 [ x ] + evectors [ [ 1 , 2 ] ] * f2 [ x ] psi2 [ x_ ] := evectors [ [ 2 , 1 ] ] * f1 [ x ] + evectors [ [ 2 , 2 ] ] * f2 [ x ] \
(* Normalize *) \
norm1 = Sqrt [ NIntegrate [ psi1 [ x ] ^ 2 , { x , - 5 , 5 } ] ] ; norm2 = Sqrt [ NIntegrate [ psi2 [ x ] ^ 2 , { x , - 5 , 5 } ] ] ; psi1n [ x_ ] := psi1 [ x ] / norm1 psi2n [ x_ ] := psi2 [ x ] / norm2 \
(* Plot wavefunctions - BLACK AND WHITE *) \
Plot [ { psi1n [ x ] , psi2n [ x ] } , { x , - 3 , 3 } , PlotStyle -> { { Black , Thick } , (* [Psi]1 : solid *) { Black , Dashed , Thick } (* [Psi]2 : dashed *) } , PlotLegends -> Placed [ LineLegend [ { Graphics [ { Black , Thick , Line [ { { 0 , 0 } , { 1 , 0 } } ] } ] , Graphics [ { Black , Dashed , Thick , Line [ { { 0 , 0 } , { 1 , 0 } } ] } ] } , { "\<[Psi]\:2081(x) Ground [solid]\>" , "\<[Psi]\:2082(x) Excited [dashed]\>" } ] , { Right , Top } ] , PlotLabel -> Style [ "\<Approximate Wavefunctions (Rayleigh-Ritz)\>" , ] , AxesLabel -> { Style [ "\<x\>" , 12 ] , Style [ "\<[Psi](x)\>" , 12 ] } , GridLines -> Automatic , GridLinesStyle -> Directive [ Gray , Dotted ] , ImageSize -> Large , Frame -> True , FrameStyle -> Black ] \
(* Plot potential and energy levels - BLACK AND WHITE *) \
Show [ (* Potential *) Plot [ g * Abs [ x ] , { x , - 3 , 3 } , PlotStyle -> { Black , Thick } , PlotRange -> { 0 , 5 } ] , (* Rayleigh - Ritz E1 *) Graphics [ { Black , Thick , Line [ { { - 3 , evalues [ [ 1 ] ] } , { 3 , evalues [ [ 1 ] ] } } ] , Text [ Style [ "\<E\:2081(RR)=\>" <> ToString [ NumberForm [ evalues [ [ 1 ] ] , 3 ] ] , 11 , ] , { - 2.3 , evalues [ [ 1 ] ] + 0.25 } ] } ] , (* Rayleigh - Ritz E2 *) Graphics [ { Black , Thick , Line [ { { - 3 , evalues [ [ 2 ] ] } , { 3 , evalues [ [ 2 ] ] } } ] , Text [ Style [ "\<E\:2082(RR)=\>" <> ToString [ NumberForm [ evalues [ [ 2 ] ] , 3 ] ] , 11 , ] , { - 2.3 , evalues [ [ 2 ] ] + 0.25 } ] } ] , (* Exact E1 *) Graphics [ { Black , Dashed , Line [ { { - 3 , exactE1 } , { 3 , exactE1 } } ] , Text [ Style [ "\<E\:2081(exact)=\>" <> ToString [ NumberForm [ exactE1 , 3 ] ] , 10 ] , { 2.0 , exactE1 - 0.25 } ] } ] , (* Exact E2 *) Graphics [ { Black , Dashed , Line [ { { - 3 , exactE2 } , { 3 , exactE2 } } ] , Text [ Style [ "\<E\:2082(exact)=\>" <> ToString [ NumberForm [ exactE2 , 3 ] ] , 10 ] , { 2.0 , exactE2 - 0.25 } ] } ] , PlotLabel -> Style [ "\<Linear Potential V(x)=g|x| with Energy Levels\>" , , 13 ] , AxesLabel -> { Style [ "\<x\>" , 12 ] , Style [ "\<Energy\>" , 12 ] } , ImageSize -> Large , Frame -> True , FrameStyle -> Black , GridLines -> Automatic , GridLinesStyle -> Directive [ Gray , Dotted ] , (* Legend *) Epilog -> { Text [ Style [ "\<Solid lines: Rayleigh-Ritz\>" , 10 ] , { - 2.2 , 4.5 } ] , Text [ Style [ "\<Dashed lines: Exact\>" , 10 ] , { - 2.2 , 4.2 } ] , Text [ Style [ "\<Thick line: Potential V(x)\>" , 10 ] , { - 2.2 , 3.9 } ] } ] \
(* === === === = STEP 10 : Comments and Analysis === === === = *) \
Print [ Style [ "\<STEP 10: Analysis and Comments\>" , , 14 ] ] Print [ ] \
Print [ "\<1. PARITY SYMMETRY:\>" ] Print [ "\< [Bullet] V(x) = g|x| is EVEN: V(-x) = V(x)\>" ] Print [ "\< [Bullet] f\:2081(x) is EVEN: f\:2081(-x) = f\:2081(x)\>" ] Print [ "\< [Bullet] f\:2082(x) is ODD: f\:2082(-x) = -f\:2082(x)\>" ] Print [ ] Print [ ] \
Print [ "\<2. BLOCK-DIAGONAL STRUCTURE:\>" ] Print [ "\< [Bullet] Hamiltonian separates into even and odd sectors\>" ] Print [ ] Print [ ] Print [ ] \
Print [ "\<3. ACCURACY OF RAYLEIGH-RITZ:\>" ] Print [ "\< [Bullet] Ground state error: \>" , NumberForm [ err1 , { 5 , 2 } ] , "\<%\>" ] Print [ "\< [Bullet] Excited state error: \>" , NumberForm [ err2 , { 5 , 2 } ] , "\<%\>" ] Print [ "\< [Bullet] Excellent for only 2 basis functions!\>" ] Print [ "\< [Bullet] RR method provides UPPER BOUNDS to true energies\>" ] Print [ ] \
Print [ "\<4. VARIATIONAL PRINCIPLE VERIFICATION:\>" ] Print [ "\< [Bullet] E\:2081(RR) [GreaterEqual] E\:2081(exact)? \>" , evalues [ [ 1 ] ] >= exactE1 ] Print [ "\< [Bullet] E\:2082(RR) [GreaterEqual] E\:2082(exact)? \>" , evalues [ [ 2 ] ] >= exactE2 ] Print [ ] Print [ ] \
Print [ "\<5. PHYSICAL INTERPRETATION:\>" ] Print [ "\< [Bullet] Linear potential g|x| is a 'V-shaped' well\>" ] Print [ ] Print [ "\< [Bullet] Ground state: concentrated near x = 0, no nodes\>" ] Print [ "\< [Bullet] Excited state: has node at x = 0 (odd parity)\>" ] Print [ ] \
Print [ "\<6. WHY THE METHOD WORKS WELL:\>" ] Print [ "\< [Bullet] Gaussian basis captures the localized nature\>" ] Print [ "\< [Bullet] Parity structure exactly preserved\>" ] Print [ "\< [Bullet] Variational freedom via linear combinations\>" ] Print [ ] \
In[4731]:= db1085e3-a732-bc42-ba4f-77b237bb4e4d
\end{lstlisting}

\medskip


\noindent\textbf{Linear Potential V(x) = g|x| - Rayleigh-Ritz Method}

\medskip


\subsection*{STEP 1: Basis Functions}

$f\:2081(x) = exp(-x^{2}/a^{2}) [EVEN function]$ $f\:2082(x) = x\[CenterDot]exp(-x^{2}/a^{2}) [ODD function]$

\begin{figure}[H]
\centering
\fbox{\parbox{0.7\textwidth}{\centering\vspace{1cm}\textit{Figure placeholder: Export figure_1.png from Mathematica}\vspace{1cm}}}
% \includegraphics[width=0.7\textwidth]{HW 8-1 pb 5-all_figures/figure_1.png}
\caption{Figure 1}
\label{fig:1}
\end{figure}

\medskip


\noindent\textbf{STEP 2: Overlap Matrix Elements S\:1d62\:2c7c = \:27e8f\:1d62|f\:2c7c\:27e9}

\medskip

$S\:2081\:2081 = \int f\:2081^{2} dx =$ S\:2081\:2082 = \int f\:2081\[CenterDot]f\:2082 dx = "\>", "", "0", 
 "", "\<" (odd integrand \rightarrow 0)


\subsection*{S\:2082\:2081 = S\:2081\:2082 =}

$S\:2082\:2082 = \int f\:2082^{2} dx =$


\subsection*{Overlap Matrix S (symbolic):}


\subsection*{Overlap Matrix S (numerical):}

$STEP 3: Kinetic Energy Matrix T\:1d62\:2c7c = -\.bd\:27e8f\:1d62|d^{2}/dx^{2}|f\:2c7c\:27e9$ $d^{2}f\:2081/dx^{2} =$ $d^{2}f\:2082/dx^{2} =$


\noindent\textbf{T\:2081\:2081 = -\.bd\int f\:2081\[CenterDot]f\:2081'' dx =}

\medskip

T\:2081\:2082 = -\.bd\int f\:2081\[CenterDot]f\:2082'' dx = 
 (parity \rightarrow 0)


\subsection*{T\:2082\:2081 = T\:2081\:2082 =}


\noindent\textbf{T\:2082\:2082 = -\.bd\int f\:2082\[CenterDot]f\:2082'' dx =}

\medskip


\subsection*{Kinetic Energy Matrix T (symbolic):}


\subsection*{Kinetic Energy Matrix T (numerical):}

STEP 4: Potential Energy Matrix V\:1d62\:2c7c = g\:27e8f\:1d62||x||f\:2c7c\:27e9 $V\:2081\:2081 = g\int |x|\[CenterDot]f\:2081^{2} dx =$ V\:2081\:2082 = g\int |x|\[CenterDot]f\:2081\[CenterDot]f\:2082 dx = 
 (parity \rightarrow 0)


\subsection*{V\:2082\:2081 = V\:2081\:2082 =}

$V\:2082\:2082 = g\int |x|\[CenterDot]f\:2082^{2} dx =$


\subsection*{Potential Energy Matrix V (symbolic):}


\subsection*{Potential Energy Matrix V (numerical):}


\subsection*{STEP 5: Hamiltonian Matrix H = T + V}


\subsection*{Hamiltonian Matrix H (symbolic):}


\subsection*{Hamiltonian Matrix H (numerical):}

*** KEY OBSERVATION: H and S are BLOCK-DIAGONAL! ***


\noindent\textbf{All off-diagonal elements are ZERO due to parity symmetry.}

\medskip

Even function f\:2081 doesn't mix with odd function f\:2082.


\subsection*{STEP 6: Solve HC = SCE}


\subsection*{Rayleigh-Ritz Energy Eigenvalues:}


\subsection*{E\:2081 = 
0.898942}


\subsection*{E\:2082 = 
2.29788}


\noindent\textbf{Eigenvectors (coefficients [c\:2081, c\:2082]):}

\medskip


\subsection*{Ground state: c = [
1.
, 
0.
]}


\subsection*{Excited state: c = [
0.
, 
1.
]}

STEP 7: Analytical Energy Formulas (Block-Diagonal) Since H and S are block-diagonal, eigenvalues are:


\subsection*{E\:2081 = H\:2081\:2081/S\:2081\:2081 =}


\subsection*{E\:2082 = H\:2082\:2082/S\:2082\:2082 =}

STEP 8: Comparison with Exact (Analytic) Results For the linear potential V(x) = g|x|, the exact solution involves Airy functions. The energy eigenvalues are: $E\:2099 = g^(2/3) \[CenterDot] |\alpha\:2099|$ $where \alpha\:2099 are the zeros of the Airy function Ai(-z).$


\noindent\textbf{Exact energy levels (from Airy function):}

\medskip


\subsection*{E\:2081(exact) = 
2.33811}


\subsection*{E\:2082(exact) = 
4.08795}


\subsection*{E\:2083(exact) = 
5.52056}


\subsection*{COMPARISON TABLE:}


\noindent\textbf{------------------------------------------------------------------------}

\medskip


\subsection*{Level Rayleigh-Ritz Exact Error (%)}


\noindent\textbf{------------------------------------------------------------------------}

\medskip


\noindent\textbf{E\:2081 0.898942 2.33811 \!\(\*RowBox[{\"\\\"61.55\\\"\"}]\)}

\medskip


\noindent\textbf{E\:2082 2.29788 4.08795 \!\(\*RowBox[{\"\\\"43.79\\\"\"}]\)}

\medskip


\noindent\textbf{------------------------------------------------------------------------}

\medskip


\subsection*{STEP 9: Visualization of Results}

\begin{figure}[H]
\centering
\fbox{\parbox{0.7\textwidth}{\centering\vspace{1cm}\textit{Figure placeholder: Export figure_2.png from Mathematica}\vspace{1cm}}}
% \includegraphics[width=0.7\textwidth]{HW 8-1 pb 5-all_figures/figure_2.png}
\caption{Figure 2}
\label{fig:2}
\end{figure}

\medskip

\begin{figure}[H]
\centering
\fbox{\parbox{0.7\textwidth}{\centering\vspace{1cm}\textit{Figure placeholder: Export figure_3.png from Mathematica}\vspace{1cm}}}
% \includegraphics[width=0.7\textwidth]{HW 8-1 pb 5-all_figures/figure_3.png}
\caption{Figure 3}
\label{fig:3}
\end{figure}

\medskip

STEP 10: Analysis and Comments


\subsection*{1. PARITY SYMMETRY:}

$\bullet V(x) = g|x| is EVEN: V(-x) = V(x)$ $\bullet \:27e8f\:2081|O|f\:2082\:27e9 = 0 for any even operator O$


\subsection*{2. BLOCK-DIAGONAL STRUCTURE:}

$\bullet Hamiltonian separates into even and odd sectors$ $\bullet Ground state: EVEN parity (only f\:2081, c\:2081\neq 0, c\:2082=0)$ $\bullet First excited: ODD parity (only f\:2082, c\:2081=0, c\:2082\neq 0)$


\subsection*{3. ACCURACY OF RAYLEIGH-RITZ:}

$\bullet Ground state error: 
61.55
%$ $\bullet Excited state error: 
43.79
%$ $\bullet Excellent for only 2 basis functions!$ $\bullet RR method provides UPPER BOUNDS to true energies$


\subsection*{4. VARIATIONAL PRINCIPLE VERIFICATION:}

$\bullet E\:2081(RR) \geq E\:2081(exact)?$ $\bullet E\:2082(RR) \geq E\:2082(exact)?$ $\bullet Approximate energies are indeed upper bounds \checkmark$


\subsection*{5. PHYSICAL INTERPRETATION:}

$\bullet Linear potential g|x| is a 'V-shaped' well$ $\bullet With g = \hbar^{2}/(ma) = 1, length scale set by a = 1$ $\bullet Ground state: concentrated near x = 0, no nodes$ $\bullet Excited state: has node at x = 0 (odd parity)$ 6. WHY THE METHOD WORKS WELL: $\bullet Parity structure exactly preserved$

\medskip

$\bullet Variational freedom via linear combinations$

\end{document}